\section{Introduction}\label{sec:introduction}


Tabular data consists of a database structured in tabular form, composed of
columns (features) and rows (observations)~\cite{yoon2020vime}. It is one of
the most commonly used data structures within a wide range of domains.
However, ML techniques developed for tabular data can be applied to any type
of data; input data, regardless of its original format, can be mapped into a
manifold, lower-dimensional abstraction of the input data and mapped back into
its original input space~\cite{kingma2019introduction, DeVries2017}.
This abstraction is often referred to as embeddings, encodings, feature
space, or latent space. In this paper, we will refer to this concept as
latent space.

Synthetic data is obtained from a generative process based on properties of
real data~\cite{assefa2020generating}. The generation of synthetic data is
essential for several objectives. For example, it is used as a form of
regularizing ML classifiers (\textit{i.e.}, data
augmentation)~\cite{wang2021regularizing}. One form of anonymizing datasets is
via the production of synthetic observations (\textit{i.e.}, synthetic data
generation)~\cite{patki2016synthetic}. In settings where only a small portion
of training data is labeled, some techniques generate artificial data using
both labeled and unlabeled data with a modified loss function to train neural
networks (\textit{i.e.}, semi-supervised learning)~\cite{laine2017temporal}.
In imbalanced learning contexts, synthetic data can be used to balance the
target classes' frequencies and reinforce the learning of minority classes
(\textit{i.e.}, oversampling)~\cite{Fonseca2021}. Some active
learning frameworks use synthetic data to improve data selection and
classifier training~\cite{Kim2021}. Other techniques employ data
generation to train neural networks without labeled data (\textit{i.e.},
self-supervised learning)~\cite{grill2020bootstrap}.

The breadth of these techniques spans multiple domains, such as facial
recognition~\cite{lv2017data}, Land Use/Land Cover
mapping~\cite{Douzas2019rs}, medical image
processing~\cite{yi2019generative}, Natural Language Processing
(NLP)~\cite{feng2021survey} or credit card default
prediction~\cite{alam2020investigation}. Finding appropriate data
generation techniques varies according to the domain and data type. In
addition, several synthetic data generation methods are specific to the
domain, data type, or target ML task. Generally, these methods rely on
the domain data's structure and are not easily transferable to tabular
data.

Overall, synthetic data generation techniques for tabular data are not as
explored as image or text data, despite their popularity and
ubiquity~\cite{fakoor2020fast}. Furthermore, these techniques are invariant to
the original data format; they can be applied to both the latent
space~\cite{DeVries2017} or tabular data. On one hand, data generation
in the latent space uses a generative model to learn a manifold,
lower-dimensional abstraction over the input
space~\cite{kingma2019introduction}. At
this level, any tabular data generation mechanism can be applied and
reconstructed into the input space if necessary. On the other hand, synthetic
data generation on tabular data can be applied to most problems. Although, the
choice of generation mechanism depends on (1) the importance of the original
statistical information and the relationships among features, (2) the target
ML task, and (3) the role synthetic data plays in the process (\textit{i.e.},
anonymization, regularization, class balancing, etc.).  For example, when
generating data to address an imbalanced learning problem (\textit{i.e.},
oversampling), the relationships between the different features are not
necessarily kept, since the goal is to reinforce the learning of the minority
class by redefining an ML classifier's decision boundaries. If the goal is to
anonymize a dataset, perform some type of descriptive task, or ensure
consistent model interpretability, statistical information must be preserved.

Depending on the context, evaluating the quality of the generated data is a
complex task. For example, for image and time series data, perceptually small
changes in the original data can lead to large changes in the Euclidean
distance~\cite{assefa2020generating, theis2016note}. The evaluation of
generative models typically accounts primarily for the performance in a
specific task, since good performance in one criterion does not imply good
performance on another~\cite{theis2016note}. However, in computationally
intensive tasks it is often impracticable to search for the optimal
configurations of generative models. To address this limitation, other
evaluation methods have been proposed to assist in this evaluation, which
typically use statistical divergence metrics, averaged distance metrics,
statistical similarity measurements, or precision/recall
metrics~\cite{chundawat2022tabsyndex, alaa2022faithful}. The relevant
performance metrics found in the literature are discussed in
Section~\ref{sec:evaluating-synthetic-data}.

\subsection{Motivation, Scope and Contributions}

We focus on data generation techniques in the tabular and latent space
(\textit{i.e.}, embedded inputs) with a focus on classification and associated
ML problems. Related literature reviews are mostly focused on specific
algorithmic or domain applications, with little to no emphasis on the core
generative process. For this reason, these techniques often appear
``sandboxed'', even though there is a significant overlap between them. There
are some related reviews published since 2019. \cite{assefa2020generating}
provides a general overview of synthetic data generation for time series data
anonymization in the finance sector. \cite{hernandez2022synthetic} reviews
data generation techniques for tabular health records anonymization.
\cite{raghunathan2021synthetic} reviews synthetic data anonymization
techniques that preserve the statistical properties of a dataset.
\cite{sauber2022use} reviews GAN-based oversampling methods for tabular
data, with a focus on cybersecurity and finance. \cite{nalepa2019data}
reviews data augmentation techniques for brain-tumor segmentation.
\cite{bayer2021survey} distinguishes augmentation techniques for text
classification into latent and data space, while providing an extensive
overview of augmentation methods within this domain. However, the taxonomy
proposed and latent space augmentation methods are not necessarily specific to
the domain. \cite{shorten2021text}, \cite{chen2021empirical},
\cite{feng2021survey} and \cite{liu2020survey} also review data augmentation
techniques for text data. \cite{sampath2021survey} reviews GAN
architectures for imbalanced learning in computer vision tasks.
\cite{yi2019generative} review Generative Adversarial Network architectures
for medical imaging. \cite{wang2020survey} reviews face data augmentation
techniques. \cite{Shorten2019}, \cite{khosla2020enhancing} and
\cite{khalifa2021comprehensive} discuss techniques for image data
augmentation.  \cite{Iwana2021} and \cite{wen2020time} also review
time series data augmentation techniques.  \cite{zhao2022graph} review data
augmentation techniques for graph data. The analysis of related literature
reviews~\footnote{Results obtained using Google Scholar, limited to articles
    published since 2019, using the search query {\fontfamily{qcr}\selectfont
        (``synthetic data generation'' OR ``oversampling'' OR ``imbalanced
        learning'' OR ``data augmentation'') AND (``literature review'' OR
``survey'')}. Retrieved on August $11^{th}$, 2022. More articles were added
later whenever found relevant.  } is shown in
Table~\ref{tab:literature-reviews}.

\begingroup\small
\setlength\LTleft{-1.5cm}
\setlength\LTright{1.5cm}
\begin{table}[t!]
    \centering
    \caption[Related literature reviews published since 2019.]{\label{tab:literature-reviews}
        Related literature reviews published since 2019. A field containing
        ``---'' indicates that the corresponding literature review does not
        focus on a particular data type, ML problem or domain.
    }
    \begin{tabularx}{\textwidth}{@{}m{.15\linewidth}lm{.15\linewidth}cX@{}}
        \toprule
        Reference & Data type & ML problem & Domain & Observations \\
        \midrule

        \cite{assefa2020generating} & --- & Data privacy &
        Finance & Analysis of applications, motivation and properties of
        synthetic data for anonymization. \\

        \cite{hernandez2022synthetic} & Tabular & Data privacy &
        Healthcare & Focus on GANs. \\

        \cite{raghunathan2021synthetic} & Tabular & Data privacy &
        Statistics & Focus on general definitions such as differential privacy
        and statistical disclosure control.\\

        \cite{sauber2022use} & Tabular & Imbalanced Learning &
        Various & Focus on oversampling with GANs in cybersecurity
        and finance.\\

        \cite{bayer2021survey} & Text & Classification & --- & Distinguish
        100 methods into 12 groups. \\

        \cite{shorten2021text} & Text & Deep Learning & --- & General
        overview of text data augmentation. \\

        \cite{chen2021empirical} & Text & Few-shot Learning & --- &
        Augmentation techniques for machine learning with limited data\\

        \cite{feng2021survey} & Text & --- & --- & Overview of augmentation
        techniques and applications on NLP tasks.\\

        \cite{liu2020survey} & Text & --- & Various & Analysis of industry
        use cases of data augmentation in NLP\@. Emphasis on input level data
        augmentation.\\

        \cite{nalepa2019data} & Image & Segmentation & Medicine & Analysis of
        algorithmic applications on a 2018 brain-tumor segmentation
        challenge.\\

        \cite{sampath2021survey} & Image & Imbalanced Learning &
        --- & Emphasis on GANs. \\

        \cite{yi2019generative} & Image & --- & Medicine & Emphasis on GANs.\\

        \cite{wang2020survey} & Image & Deep Learning & --- & Regularization
        techniques using facial image data. Emphasis on Deep Learning
        generative models.\\

        \cite{Shorten2019} & Image & Deep Learning & --- & Emphasis on
        data augmentation as a regularization technique.\\

        \cite{khosla2020enhancing} & Image & --- & --- & Broad overview of
        image data augmentation. Emphasis on traditional approaches. \\

        \cite{khalifa2021comprehensive} & Image & --- & Various & General
        overview of image data augmentation and relevant domains of
        application.\\

        \cite{Iwana2021} & Time series & Classification & --- &
        Defined a taxonomy for time series data augmentation.\\

        \cite{wen2020time} & Time series & Various & --- & Analysis of data
        augmentation methods for classification, anomaly detection and
        forecasting.\\

        \cite{zhao2022graph} & Graph & Various & --- & Graph data
        augmentation for supervised and self-supervised learning.\\

        \bottomrule
        
    \end{tabularx}
\end{table}
\endgroup

This literature review focuses on generation mechanisms applied to tabular
data across the main ML techniques where tabular synthetic data is used.  We
also discuss generation mechanisms used in the latent space, since the
generation mechanisms in tabular data and latent space may be used
interchangeably. In addition, we focus on the ML perspective of synthetic
data, as opposed to the practical perspective; according to the practical
perspective, synthetic data is used as a proxy of real data when it is
inaccessible, essential, and a secondary asset for tasks like education,
software development, or systems demonstrations~\cite{mannino2019real}.
The ML perspective focuses on the generation of synthetic data based on
existing, naturally occurring data to either improve a ML task or replace the
original data. 

The different taxonomies of synthetic data generation established in the
literature follow a similar philosophy but vary in terminology and are often
specific to the technique discussed. Regardless, it is possible to establish a
broader taxonomy without giving up on specificity. This study provides a joint
overview of the different data generation approaches, domains, and ML
techniques where data generation is being used, as well as a common taxonomy
across domains. It extends the analyses found in these articles and uses the
compiled knowledge to identify research gaps. We compare the strengths and
weaknesses of the models developed within each of these fields. Finally, we
identify possible future research directions to address some of the
limitations found. The main contributions of this paper are summarized below:

\begin{itemize}

    \item Bridge different ML concepts that use synthetic data generation
        techniques (Sections~\ref{sec:background}
        and~\ref{sec:algorithmic-applications});

    \item Propose a synthetic data generation/data augmentation taxonomy to
        address the ambiguity in the various taxonomies proposed in the
        literature (Section~\ref{sec:taxonomy});

    \item Characterize all the relevant data generation methods using the
        proposed taxonomy (Sections~\ref{sec:taxonomy}
        and~\ref{sec:algorithmic-applications});

    \item Consolidate the current generation mechanisms across the
        different techniques and methods to evaluate the quality of
        synthetic data generation (Sections~\ref{sec:generation-mechanisms}
        and~\ref{sec:evaluating-synthetic-data});

    \item Highlight the main challenges of synthetic data generation and
        discuss possible future research directions
        (Sections~\ref{sec:discussion} and~\ref{sec:future-work}).

\end{itemize}

\subsection{Bibliometric Data Collection}

Considering the goals determined in this study, the literature collection is
more complex than usual; the wide range of domains and ML problems where
synthetic data is used involved considering different naming conventions for
the same concepts. In addition, it involved the identification of such domains
and ML problems and checking for less popular mechanism combinations (some of
which do not show up in any general query, unless purposefully looked up). To
achieve this, we followed a 2-step approach:

\begin{enumerate}

    \item Collection of related literature reviews/surveys/systematic studies:
        Allowed us to understand which domains and ML problems to discuss,
        naming conventions (for example, latent vs. embeddings vs. econdings
        vs. feature space; or synthetic vs. anonymized vs. augmented vs.
        artificial vs. resampled data), differences in taxonomies across
        domains, and importance of this study.  Based on the rate at which
        research is being developed in ML, we considered exclusively studies
        from 2019 onward since any literature review prior to this date can be
        deemed outdated and overlapping with more recent literature reviews.

    \item Individual queries according to specific domains, ML problems,
        concepts, and taxonomy proposed. The large amount of queries performed
        at this stage involved a case-by-case screening of the studies’
        relevancy, searching through the bibliographies of existing papers as
        well as papers citing the original study and the inclusion of studies
        that were a priori known by the authors. 

\end{enumerate}

The studies included in this literature review were collected from Google
Scholar. Compared to other sources, such as Scopus, Web of Science, Dimensions
or OpenCitation’s COCI, several studies found Google Scholar to be the most
complete source for literature search. According
to~\cite{martin2021google},
it contains 88\% of all citations, many of which are not found in other
sources, and contains 89\% to 94\% of the citations found by the remaining
sources. Another study found even higher
disparities~\cite{martin2018google};
Google Scholar found 93\% to 96\% of citations across all areas, far more
complete than the remaining options, and found 95\% and 92\% of Web of
Science's and Scopus' citations, respectively. Since a large percentage of the
journals/repositories considered are high-impact journals, conference
proceedings, or well-known repositories, it is reasonable to assume all the
targeted studies were readily available via Google Scholar.


\subsection{Paper Organization}

The rest of this paper is organized as follows: Section~\ref{sec:background}
defines and formalizes the different concepts, goals, trade-offs, and
motivations related to synthetic data generation. Section~\ref{sec:taxonomy}
defines the taxonomy used to categorize all the algorithms analyzed in this
study. Section~\ref{sec:algorithmic-applications} analyses all the algorithms
using synthetic data generation, distinguished by learning problem.
Section~\ref{sec:generation-mechanisms} describes the main generation
mechanisms found, distinguished by generation type.
Section~\ref{sec:evaluating-synthetic-data} reviews performance evaluation
methods of synthetic data generation mechanisms. Section~\ref{sec:discussion}
summarizes the main findings and general recommendations for good practices on
synthetic data usage. Section~\ref{sec:future-work} discusses limitations,
research gaps, and future research directions. Section~\ref{sec:conclusions}
presents the main conclusions drawn from this study.
