\chapter{Conclusion}~\label{chp:conclusion}

This dissertation focused on addressing 4 main research questions. The main
focus of these questions was to address common scenarios where typical ML
techniques will not work as intended: (1) imbalanced learning and (2)
supervised learning with scarcity of labeled data. As a result, the main
contributions of this dissertation are two-fold: (1) an oversampling technique
to address the limitation of oversampling on datasets with mixed data types
and (2) an AL framework that relies on synthetic data to reduce data
collection requirements to produce well-performing ML classifiers.

In Chapter~\ref{chp:synthetic-data-review} we studied the state-of-the-art in
data augmentation algorithms, which was necessary to proceed to subsequent
steps of the work plan. The remaining work presented in this dissertation was
developed based on these findings. The Geometric-SMOTENC oversampler proposed
in Chapter~\ref{chp:gsmotenc} uses the generation mechanism described
in~\cite{Douzas2019}, while encoding the continuous features, calculating the
selected observations' nearest neighbors and generating the categorical
feature values for the synthetic data using the method described
in~\cite{Chawla2002}. In Chapter~\ref{chp:kmeans-smote} we used an
oversampling method to address the prevailing problem of imbalanced learning
in LULC\@.

In Chapters~\ref{chp:al-generator-lulc}
and~\ref{chp:active-learning-augmentation} we proposed a modification of the
AL framework as a means to address RQ4. However, the work presented can be
further enriched with the application of other methods that leverage
information from unlabeled data, specifically semi-supervised and
self-supervised learning techniques to form a single framework.

Despite the focus on the case of LULC classification, which is particularly
challenging due to its dimensionality, all of the methods proposed are
generalizable. In addition, they achieved a statistically significant superior
performance compared to the state-of-the-art. Overall, the entirety of the
work presented is fully replicable, open source, and domain transferable. 
