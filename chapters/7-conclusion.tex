\chapter{Conclusions}~\label{chp:conclusion}

This dissertation focused on addressing 4 main research questions. The main
focus of these questions was to address common scenarios where typical ML
techniques will not work as intended: (1) imbalanced learning and (2)
supervised learning with scarcity of labeled data. As a result, the main
contributions of this dissertation are two-fold: (1) an oversampling technique
to address the limitation of oversampling on datasets with mixed data types
and (2) an AL framework that relies on synthetic data to reduce data
collection requirements to produce well-performing ML classifiers.

In Chapter~\ref{chp:synthetic-data-review} we studied the state-of-the-art of
synthetic data generation techniques, which was necessary to proceed to subsequent
steps of the work plan. We found several limitations in the literature
regarding latent space learning, selection of generation mechanisms, data
privacy mechanisms, analysis of quality of synthetic data for regularization
techniques, consistency and interpretability of generative neural network
methods, ensmble techniques for tabular data, oversampling tabular data with
mixed data types, lack of research of synthetic data generation in tabular few-shot
learning and lack of research of the effect of synthetic data towards model
fairness and bias.

The remaining work presented in this dissertation was developed based on these
findings. The Geometric-SMOTENC oversampler proposed in
Chapter~\ref{chp:gsmotenc} uses the generation mechanism described
in~\cite{Douzas2019}, while encoding the continuous features, calculating the
selected observations' nearest neighbors and generating the categorical
feature values for the synthetic data using the method described
in~\cite{Chawla2002}. This method can be considered a generalization of the
classical SMOTENC approach, since a specific parametrization of this algorithm
will replicate SMOTENC's behavior. However, this method allows a significantly
wider array of possibilities and high variability in the synthetic data being
generated. In addition, it may be applied before any type of categorical
feature encoding.

In Chapter~\ref{chp:kmeans-smote} we used an oversampling method to address
the prevailing problem of imbalanced learning in LULC\@. A distinctive
characteristic of LULC classification is the potential for some classes to
contain significantly different spectral signatures (\textit{e.g.}, two
patches of forests or agricultural areas may contain entirely different types
of vegetation, despite having the same class). In this case, clustering-based
synthetic data generation assists in distinguishing these differences
within a minority class among clusters and avoid the generation of noisy
synthetic data.

In Chapter~\ref{chp:al-generator-lulc} we introduce a modification of the
typical AL framework in order to address RQ4. To the best of our knowledge,
this was one of the first methods to implement synthetic data into AL using
tabular data. The proposed framework showed a significant reduction of the
amount of required labeled data to reach a given performance threshold.
Consequently, this method may be used to reduce the labeling cost when
preparing a training dataset without harming classification performance\@.

In Chapter~\ref{chp:active-learning-augmentation} we focus on the
generalization of the AL framework previously proposed regarding both the
domain of application and data generation policy. We introduce a new component
to this framework to optimize the data generation method within each
iteration, broaden the data generation policies employed, and test the new
framework across several datasets from different domains. Overall, this
optimized method to employ synthetic data in AL further reduced the amount of
data labeling required to achieve the same classification performance. In
addition, we also found a significant improvement in classification
performance using this approach, even when compared to a classifier trained
using the fully labeled dataset. The work presented in this dissertation can
be further enriched with the application of other methods that leverage
information from unlabeled data, specifically semi-supervised and
self-supervised learning techniques to form a single framework.

Although this dissertation addressed some important open questions found in
the literature, several others must be explored in future work;
Chapter~\ref{chp:synthetic-data-review} discusses such limitations and
research gaps into a high level of detail. Despite the focus on the case of
LULC classification, which is particularly challenging due to its
high-dimensionality, all of the methods proposed are entirely generalizable.
In addition, all of the methods proposed/explored achieved a statistically
significant superior performance compared to the state-of-the-art. Overall,
the entirety of the work presented is fully replicable, open source, and
domain transferable. 
